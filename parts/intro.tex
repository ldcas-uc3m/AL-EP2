\part{Introducción}
A comienzos del siglo XXI, el entorno digital, en particular la web, comenzó a jugar un papel esencial en la vida cotidiana, consolidándose como una herramienta fundamental para la difusión de información y la comunicación global. En España, este proceso coincidió con la transición de la peseta al euro y una notable expansión de la banda ancha, lo que permitió a los usuarios acceder a internet con mayor rapidez y facilidad, fomentando la proliferación de servicios en línea.

En este contexto, los servicios de la sociedad de la información y el comercio electrónico empezaron a establecerse como elementos clave en el ámbito empresarial, generando nuevas oportunidades de negocio en la red. La empresa y restaurante \textit{Loft76}, consciente de estas tendencias, desarrolló su página web para facilitar el acceso de los usuarios a su carta, localización y servicios.

El presente análisis examina la situación de la web de \textit{Loft76} en relación con las obligaciones establecidas por la Ley 34/2002, de 11 de julio, de Servicios de la Sociedad de la Información y de Comercio Electrónico \cite{LSSICE} (en adelante, ``LSSICE''), que regula los deberes legales de las empresas que operan en internet. Este estudio se enfoca en los aspectos más relevantes de la normativa, destacando los requisitos de transparencia, protección de datos y condiciones de uso, con el fin de evaluar el cumplimiento normativo de la empresa en su plataforma en línea.
