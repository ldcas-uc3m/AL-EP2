\part{Conclusiones}
Este proyecto ha sido una experiencia muy enriquecedora, porque ha permitido aplicar y trasladar los conocimientos adquiridos en el análisis de la LSSICE, a un entorno real.
A diferencia de la falta de información y ambigüedad de la normativa pasada Ley Orgánica 3/2018, de 5 de diciembre, de Protección de Datos Personales y garantía de los derechos digitales \cite{LOPDPGDD}, ha sido más sencillo comprender e interpretar el escrito, las sanciones y aplicar la normativa a un caso real.
Los comercios deben recordar, que no basta con ofrecer eficiencia, calidad y buenos precios, sino que es necesario cumplir con la normativa vigente, y en este caso, la LSSICE.

El incumplimiento normativo, ya sea por desconocimiento o por desinterés, puede acarrear sanciones económicas y pérdida de reputación, lo que puede llevar a la empresa a la quiebra.
Puede que hace unos años, internet no tuviera tanta importancia, pero en la actualidad, con la digitalización de la sociedad, es imprescindible cumplir con la normativa, y aún más, en el sector de los servicios.

Es alarmante que, en muchos casos, la prioridad se otorgue a la eficiencia y la calidad del servicio en detrimento de la seguridad y la privacidad. Esto resalta la urgente necesidad de fomentar la conciencia y la formación en el ámbito normativo digital. Cumplir con la LSSICE no solo es un requisito legal, sino también un compromiso con la ética empresarial y la confianza de los usuarios.