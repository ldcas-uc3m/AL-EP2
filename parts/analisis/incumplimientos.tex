\section{Presuntos incumplimientos}\label{sec:incumplimientos}
% DCE art 2.a
% 2.1, 5
% 10.1

La Directiva 98/48/CE del Parlamento Europeo y del Consejo de 20 de julio de 1998 \cite{D98/48/CE}, que modifica la Directiva 98/34/CE \cite{DCE} por la que se establece un procedimiento de información en materia de las normas y reglamentaciones técnicas, define en el artículo 2.2 los servicios de la sociedad de la información como \textquote{todo servicio prestado normalmente a cambio de una remuneración, a distancia, por vía electrónica y a petición individual de un destinatario de servicios} y a un prestador de servicios como \textquote{cualquier persona física o jurídica que suministre un servicio de la sociedad de la información}.

Se considera que la empresa \textit{Loft76} presta un servicio de la sociedad de la información, dado que su página web supone una actividad económica, la cual es publicitar su actividad económica de restauración. Según el artículo 10.2 de la LSSICE, \textquote{La obligación de facilitar esta información se dará por cumplida si el prestador la incluye en su página o sitio de Internet en las condiciones señaladas}.

La LSSICE en su artículo 2.1 establece que \textquote{[e]sta Ley será de aplicación a los prestadores de servicios de la sociedad de la información establecidos en España y a los servicios prestados por ellos}. La empresa \textit{Loft76}, al estar prestando servicios en la localidad de Guadalajara, y prestar un servicio no excluido por el artículo 5 de la misma, está sujeta a las obligaciones de la sección 1ª del capítulo II.

En concreto, según el artículo 10.1, \textquote{el prestador de servicios de la sociedad de la información estará obligado a disponer de los medios que permitan, tanto a los destinatarios del servicio como a los órganos competentes, acceder por medios electrónicos, de forma permanente, fácil, directa y gratuita} (apartado \textit{a}) \textquote{[s]u nombre o denominación social; su residencia o domicilio [\dots]; su dirección de correo}, (apartado \textit{b}) \textquote{[l]os datos de su inscripción en el Registro Mercantil} y (apartado \textit{e}) \textquote{[e]l número de identificación fiscal que le corresponda}. El apartado \textit{f} también especifica que \textquote{[c]ando el servicio de la sociedad de la información haga referencia a precios, se facilitará información clara y exacta sobre el precio del producto o servicio, indicando si incluye o no los impuestos aplicables}.

En la página web de la empresa \textit{Loft76} \cite{loft76} aparece exclusivamente una dirección postal y electrónica, y un teléfono de contacto de la empresa (\figref{contacto}), pero ninguna de la otra información requerida por los apartados \textit{a}, \textit{b}, y \textit{e} del artículo mencionado. Con respecto al apartado \textit{f}, la página sí que incluye unos precios orientativos sobre sus servicios (\figref{precios}) y su carta (\figref{carta}), pero no especifica si ese precio incluye impuestos, o si son aplicables o no. También es cierto que estos precios pueden no ser fijos, y que existe una forma gratuita de consultar estos precios a través de un formulario de contacto (\figref{formulario}).

\rasterfigure[.8]{contacto}{Información de contacto de Loft76}

\rasterfigure[.8]{precios}{Información sobre precios de servicios de Loft76}

\rasterfigure[.8]{formulario}{Formulario de contacto de Loft76}

\rasterfigure[.8]{carta}{Carta de precios de Loft76}

