\section{Posibles sanciones}
% 38, 39

Según el artículo 38.3, apartado \textit{a} de la LSSICE, se consideran \textquote{infracciones graves} \textquote{[e]l incumplimiento significativo de lo establecido en los párrafos a) y f) del artículo 10.1}. También se consideran \textquote{infracciones leves} según el artículo 38.4, apartado \textit{b}, \textquote{[n]o informar en la forma prescrita por el artículo 10.1 sobre los aspectos señalados en los párrafos b), c), d), e) y g) del mismo, o en los párrafos a) y f) cuando no constituya infracción grave}. Como hemos mencionado en la \secref{incumplimientos}, la empresa \textit{Loft76} incumple los apartados \textit{a}, \textit{b}, \textit{e}, y \textit{f} por lo que incurre en infracciones graves y leves.

El artículo 39.1 establece las sanciones por las infracciones, por la \textquote{comisión de infracciones graves, multa de 30.001 hasta 150.000 euros} (apartado \textit{b}), y por la \textquote{comisión de infracciones leves, multa de hasta 30.000 euros} (apartado \textit{c}).

Sin embargo, dada la naturaleza de estas presuntas infracciones, es posible que se pueda apercibir a la empresa. Según el artículo 39 ter., \textquote{Los órganos con competencia sancionadora [\dots] podrán acordar no iniciar la apertura del procedimiento sancionador y, en su lugar, apercibir al sujeto responsable, a fin de que en el plazo que el órgano sancionador determine, acredite la adopción de las medidas correctoras que, en cada caso, resulten pertinentes, siempre que los hechos fuesen constitutivos de infracción leve o grave conforme a lo dispuesto en esta Ley}.